\documentclass[a4paper, 11pt]{article} %option para: draft mode not insert figure, give a faster preview
\usepackage[UTF8]{ctex} %for chinese
\usepackage{amsmath}    %for math
\usepackage{geometry}   %for page setting
\usepackage{graphicx}   % for insert graph
\usepackage{color}    %for link and code color
\usepackage{float}    %for graph table in the follow word
\usepackage[colorlinks,linkcolor=blue,anchorcolor=blue,citecolor=green]{hyperref} %for a linked ref

\setlength{\parindent}{2em} % maybe not use delete it ?
\geometry{left=3.0cm, right=3.0cm, top=3.0cm, bottom=3.0cm}

\graphicspath{{figure/}}


%%%%%%%%%%%%%%%%%%%%%%%%%%%%%%%%%%%%%%%%%%%%%%%%%%%%%%%%%%%%%%%%%%%%%%%%%%%%%%%%%%%%%%%%%%%%%%
%%%%             optional package default in comment to improve compile speed            %%%%

% \usepackage{physics}    %for a readable formulation


% \usepackage[all]{hypcap} % use to jump to the top of figure/table rather than merely caption

% \usepackage{fancyhdr}
% \pagestyle{fancy} % leftmark is a build-in marco means the current higher level in markboth, right is a lower one

% \addtolength{\headheight}{\baselineskip} % use to delete headheight waring
% \fancyfoot[C]{\thepage}
% \fancyhead[L]{\leftmark}
% \fancyhead[R]{\rightmark}
% \renewcommand{\headrulewidth}{0pt}
% \pagestyle{fancy}

% \usepackage[final]{listings} %for code use final to exclude draft mode to show whenever it is
% \definecolor{dkgreen}{rgb}{0,0.6,0}
% \definecolor{gray}{rgb}{0.5,0.5,0.5}
% \definecolor{mauve}{rgb}{0.58,0,0.82}

% \lstset{frame=tb,
%   language=c++,
%   aboveskip=3mm,
%   belowskip=3mm,
%   showstringspaces=false,
%   columns=flexible,
%   basicstyle={\small\ttfamily},
%   numbers=left, %none, right
%   numberstyle=\tiny\color{gray},
%   keywordstyle=\color{blue},
%   commentstyle=\color{dkgreen},
%   stringstyle=\color{mauve},
%   breaklines=true,
%   breakatwhitespace=true,
%   tabsize=2,
%   captionpos=b
% }
% \renewcommand{\lstlistingname}{源代码} % to change the prefix as 源码 not Listing
% \renewcommand{\lstlistlistingname}{源代码} % header name in list of listing
% % \usepackage{fontspec}
% % \setmonofont{Consolas} % set consolas in coding box, ONLY XeLaTeX so default comment remember a fontspec only used at the following


% \usepackage{wrapfig}  % for picutre on the paragraph right, cannot work with chinese par in pdflatex


% % \usepackage{fontspec} % for other font family


% \usepackage{multirow} %for group row in a table


% % Threeparttable
% \usepackage{threeparttable}
% \usepackage{booktabs}


% % drawing script tikz and Circuitlib 
% \usepackage{tikz}
% \usetikzlibrary{circuits.ee.IEC}
% \usetikzlibrary{positioning}

% \usepackage[american]{circuitikz}

%%%%%%%%%%%%%%%%%%%%%%%%%%%%%%%%%%%%%%%%%%%%%%%%%%%%%%%%%%%%%%%%%%%%%%%%%%%%%%%%%%%%%%%
%%%%%                         new recommand setting area                           %%%%

% % Count those in subsection
% \makeatletter
% \@addtoreset{equation}{subsection}
% \@addtoreset{figure}{subsection}
% \@addtoreset{table}{subsection}
% \makeatother
% \renewcommand {\thefigure} {\thesubsection{}.\arabic{figure}}
% \renewcommand {\thetable} {\thesubsection{}.\arabic{table}}
% \renewcommand {\theequation} {\thesubsection{}.\arabic{equation}}


% \newcommand{\parallelsum}{\mathbin{\!/\mkern-5mu/\!}} % parallesum for resistor
% \newcommand{\upcite}[1]{\textsuperscript{\textsuperscript{\cite{#1}}}}

%%%%%%%%%%%%%%%%%%%%%%%%%%%%%%%%%%%%%%%%%%%%%%%%%%%%%%%%%%%%%%%%%%%%%%%%%%%%%%%%%%%%%%%
%%%%%                         introduction section area                           %%%%


\title{\huge{\textbf{实验x yyy} \\ \textbf{实验报告}}}
\author{
    \\
    \\
    \\
    \\
    \\
    \\
    \begin{tabular}{ll}
        班级: & 自72\\
        姓名:& 吴文绪\\
        学号: &2017010910\\
    \end{tabular}
}
\date{实验时间\quad y年 m月 d日}

\begin{document}


\maketitle
\thispagestyle{empty}
\setcounter{page}{0}
\newpage

% \tableofcontents
% \thispagestyle{empty}
% \setcounter{page}{0}
% \newpage

\section{实验目的}


\section{预习任务}


\section{实验原理}


\section{数据分析}


\section{思考题}

\section{实验总结}

% \newpage
% \section{常用格式待查}

% \subsection{文本编辑}
% \paragraph{}
% \textbf{加粗}\footnote{脚注}\textit{斜体}
% \vspace{1cm}%水平间距调整
% \vskip 7cm %垂直间距调整


% \textbf{加粗}
% \footnote{footnote}
% \par{} 强制段落和缩进测试,人类的本质是复读机,人类的本质是复读机,人类的本质是复读机,人类的本质是复读机
% \par{} 强制段落和缩进测试par用来缩进,人类的本质是复读机,人类的本质是复读机,人类的本质是复读机,人类的本质是复读机\S chapter
% \paragraph{paragraph用来加这个粗字} 强制段落和缩进测试pargraph没有,人类的本质是复读机,人类的本质是复读机,人类的本质是复读机,人类的本质是复读机

% \subsection{图}

% \begin{figure}[H]
%     \centering
%     \includegraphics[width = 0.8\textwidth]{test.png}
%     \caption{标题} %最终文档中希望显示的图片标题
%     \label{test}
% \end{figure}

% \par{} 引用测试 图\ref{test}

% \begin{figure}[H]
%   \centering
%   \begin{minipage}[H]{0.48\textwidth}
%     \centering
%     \includegraphics[width = 0.9\textwidth]{test.png}
%     \caption{标题1}
%   \end{minipage}
%   \begin{minipage}[H]{0.48\textwidth}
%     \centering
%     \includegraphics[width = 0.9\textwidth]{test.png}
%     \caption{标题2}
%   \end{minipage}
% \end{figure}

% 正常的段落

% \begin{wrapfigure}{r}{0pt}    
%     \includegraphics[width = 0.4\textwidth]{test.png}
%     \caption{right hand side}
% \end{wrapfigure}

% 我就正常写个段落,他会在左边。我就正常写个段落,他会在左边。我就正常写个段落,他会在左边。我就正常写个段落,他会在左边。我就正常写个段落,他会在左边。我就正常写个段落,他会在左边。我就正常写个段落,他会在左边。我就正常写个段落,他会在左边。我就正常写个段落,他会在左边。我就正常写个段落,他会在左边。我就正常写个段落,他会在左边。我就正常写个段落,他会在左边。我就正常写个段落,他会在左边。我就正常写个段落,他会在左边。我就正常写个段落,他会在左边。我就正常写个段落,他会在左边。我就正常写个段落,他会在左边。我就正常写个段落,他会在左边。我就正常写个段落,他会在左边。我就正常写个段落,他会在左边。

% \newpage
% \subsection{数学}
% \par{} 引用测试 \ref{test}

% \paragraph{} the theorem is named after Russian Al. In this variant of the \textbf{CLT} the random $X_i + \sigma$

% \paragraph{} Suppose ${X_1,X_2,...}$ is a sequence of independent random variables, each with finite expected value $\mu_i$ and variance $\sigma^2$. Define
% \begin{equation}
%    s_n^2=\sum_{i=1}^n \sigma_i^2 \qquad i = 1,2, \cdots n
% \end{equation}

% \begin{equation*}
%   \int_{a}^{b}f(x) \,\mathrm{d}x = \frac{\mathrm{d}p}{\mathrm{d}q}
% \end{equation*}

% \begin{align*}
%     P(X(S_n)=k-1|X(S_{n-1})=k)&=1\\
%     P(X(S_n)=i,i\neq k-1|X(S_{n-1})=k)&=0
% \end{align*}

% \begin{equation}
%     P=\left[ \begin{array}{ccccc}
%          0&0&\cdots&0&1  \\
%          1&0&\cdots&0&0 \\
%          0&1&\cdots&0&0\\
%          \vdots&\vdots&\ddots&\vdots&\vdots\\
%          0&0&\cdots&1&0
%     \end{array} \right]
% \end{equation}

% % physics package to get a more readable formulation but cannot get the hold preview

% \begin{equation*}
%   \dv{f}{x} = \pdv[n]{g}{t} = \pdv{f}{x}{y} = \mqty(a & b \\ c & d) = \int_{a}^{b}\sin[2](x) \dd x  = \abs{a} \qq{quick insert word}
% \end{equation*}

% \subsection{表}
% % use latex table tool-- an online generator tool to fast build a table
%   \begin{table}[H]
%     \centering
%       \begin{tabular}{cc}
%           A & a\\
%           BBB & b\\
%       \end{tabular}
%     \caption{test}
%   \end{table}

%   \begin{table}[H]
%     \centering
%     \begin{tabular}{||c|c|c|c||}
%       \hline
%       \multirow{2}*{合并行}&\multicolumn{3}{c||}{合并列}\\
%       \cline{2-4}
%       &测试&测试&测试\\
%       \hline
%       \end{tabular}
%     \caption{muti table}
%   \end{table}

%   \begin{table}[H]
%     \centering
%     \begin{threeparttable}
%       \small 
%       \begin{tabular} {llll}
%         \toprule
%         GDP in base year(2010)/billion\$ & $A/A'$ & $S/S'$ & $G/G'$ \\
%         \midrule
%         123.166  & 2.187 & 1.129 & 1.471 \\
%         \bottomrule
%       \end{tabular}
%       \caption{Vietnam economic indicators}
%     \end{threeparttable}
%   \end{table}



% \subsection{代码}

% \begin{lstlisting}[caption = c++ hellow world]
% #include<iostream>
% int main() {
%     int a;
%     for (int i = 0; i < 100; i++) {
%       if (i % 2 == 0) {
%         printf("Hello World!");
%       }
%     }
%   return 0;
% }
% \end{lstlisting}

% % use [language to change the para in lstset] to 
% \begin{lstlisting}[language = python, caption = py hellow world]
% for c in 'Python'
%     print(c + '!')
% \end{lstlisting}

% \subsection{条目列表}

% \begin{itemize}
%   \item[*]INIT$\to$START: 按下START
%   \item[*]INIT$\to$INIT: 未按下START 
%   \item[*]START$\to$INPUT: START状态用于将输出的money, timer置为0,故此次态必为INPUT
%   \item[*]INPUT$\to$INPUT: 没有按下ok,继续输入数字
% \end{itemize}

% \subsection{制图}
% % may be warning incomptiable color with lstlisting ,recommand compile figure alone and insert to avoid it and speed up document compile
% \begin{figure}[H]
% \centering 
% \begin{tikzpicture}[scale = 0.8] % or width with textwidth
%   \draw[help lines] (8, 8) grid(11,11);
%   \draw [->] (8, 8) --(9, 11);
%   \draw [<->][line width = 2][black][dashed] (8,10) -- (8,8) -- (11,8);
%   \draw[blue] (11,12) rectangle(10, 11);
%   \draw[blue, fill = orange] (10,13) circle[radius = 0.1];
%   \node[below] at (10, 13) {label $m \alpha th$};
%   \draw [thick, black] (8,8) to [out=90,in=180] (9,9) to [out=0,in=180] (10.5,8) to [out=0,in=-135] (12,9) node[right, black]{in out degree curve};
%   \node at (12,13) {free label $m \alpha th$};
  
%   \path [fill=yellow] (0,0) -- (0,5) to [out=-80, in=160] (3,.8) -- (3,0) -- (0,0);
%   \draw [<->] (0,6) node [left] {$P$} -- (0,0) node [below left] {(0,0)} -- (7,0) node [below] {$Q$};
%   \draw [ultra thick, dashed] (0,.8) node [left] {$P^*=.8$} -- (3,.8) -- (3,0) node [below] {$Q^*=3$};
%   \draw [fill] (3,.8) circle [radius=.1];
%   \draw [thick] (0,5) to [out=-80, in=160] (3,.8) to [out=-20, in=175] (6,0);     
%   \end{tikzpicture}
%   \caption{tikz制图供求曲线以及右上角的杂图}
% \end{figure}

% \begin{tikzpicture}
%   \draw
%   (0, 2) node[and port] (myand1) {}
%   (0, 0) node[and port] (myand2) {}
%   (2, 1) node[xnor port] (myxnor) {}
%   (myand1.out) -| (myxnor.in 1)
%   (myand2.out) -| (myxnor.in 2);
% \end{tikzpicture}

% \begin{figure}[H]
%     \centering
%     \begin{tikzpicture}
%       \draw
%       (4, 0) node[rground] (ground){}
%       (4, 3) node[npn] (npn) {}
%       (4, 6) node[rground, rotate = 180] (vcc) {} (4, 6) node[right] {$V_{cc}$}
%       (0, 3) to[C, l_=$C_1$,  v^<=$U_{BEQ}$] (2, 3)
%       (0, 3) node[] (C1n) {} (2, 3) node[] (C1p) {}
%       (4, 4) to[C, l_=$C_2$,  v^>=$U_{CEQ}$] (6, 4)
%       (4, 4) node[] (C2p) {} (6, 4) node[] (C2n) {}
%       (3, 5) to[R, l=$R_b$] (3, 3)
%       (3, 5) node[] (Rbp) {} (3, 3) node[] (Rbn) {}
%       (4, 4) to[R, l=$R_c$] (4, 6)
%       (4, 4) node[] (Rcn) {}  (4, 6) node[] (Rcp) {}
%       (6, 1) to[R, l=$R_L$] (6, 3)
%       (6, 1) node[] (RLn) {}  (6, 3) node[] (RLp) {}
%       ;
%       \draw
%       (C1n) to[short, o-] (C1n)
%       (C2n) to[short, o-] (C2n)
%       (Rbn) to[short, *-] (npn.B)
%       (C1p) to[short, -*] (Rbn)
%       (Rbp) to[short] (3, 6) to[short, -*] (vcc)
%       (Rcp) to[short, -*] (vcc)
%       (npn.C) to[short, -*] (Rcn)
%       (0, 0) to[short, o-*] (ground) to[short, *-o] (6, 0)
%       (ground) to[short, *-] (npn.E)
%       ;
%       \draw[dashed]
%       (6, 0) to[short] (RLn)
%       (C2n) to[short] (RLp)
%       ;
%       \draw
%       (C1n) node[below] {$+$} (0, 0) node[above] {$-$}
%       (C2n) node[right] {$+$} (6, 0) node[right] {$-$}
%       (0, 1.5) node[] {$u_i$}
%       (6.2, 2) node[right] {$u_o$}
%       ;
% \end{tikzpicture}
%     \caption{amplifer circuit(by circuitikz)}
% \end{figure}

% \newpage
% \appendix % into appendix mode, section will count as ABCD

% \section{图表源码索引}

% \listoffigures
% \listoftables
% \lstlistoflistings % code listing

\end{document}
